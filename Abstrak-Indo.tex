\chapter*{Abstrak}

Keadaan darurat adalah peristiwa yang dapat mengancam dan mengganggu kehidupan seseorang. Keadaan darurat terjadi secara tiba-tiba dengan tempat dan waktu yang tak terduga. Banyak dampak fatal dalam keadaan darurat dapat dicegah atau dikurangi dengan bantuan pertama yang memadai. Sebagian besar negara telah menerapkan beberapa bentuk layanan darurat untuk melakukan pertolongan pertama dalam situasi darurat. Di kota Bandung, layanan darurat menggunakan nomor telepon 112 sebagai panggilan darurat tunggal dan dikelola oleh operator. Operator ini harus memutuskan unit mana yang terdekat dengan lokasi darurat. Lingkungan mereka beroperasi ditandai dengan tingkat ketidakpastian yang tinggi, karena mereka tidak mengetahui jarak yang pasti dari masing-masing unit ke lokasi darurat. K-Nearest Neighbor (KNN) dapat digunakan untuk menemukan unit gawat darurat terdekat. Namun, dengan tingginya jumlah panggilan darurat di kota Bandung, KNN memiliki biaya komputasi yang tinggi. KNN harus mengukur setiap jarak unit gawat darurat ke lokasi darurat berulang-ulang. Dengan mengggunakan bantuan Network Voronoi Diagram (NVD), peran KNN dapat dioptimalkan untuk menemukan unit gawat darurat terdekat dengan mempartisi ruang ke dalam wilayah untuk masing-masing unit. Jadi NVD bisa langsung mendapatkan unit terdekat tersebut dengan melihat ke dalam lokasi darurat wilayah unit mana yang ada.
  
\vspace{0.5 cm}
\begin{flushleft}
{\textbf{Kata Kunci:} Darurat, Pertolongan Pertama, NVD, KNN.}
\end{flushleft}