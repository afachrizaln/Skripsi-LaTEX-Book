\chapter*{Abstract}

Emergency is an event that can threaten and disrupt a person's life. Emergency situation occur suddenly with unexpected place and time. Many fatal impact in emergency can be prevented or reduced by adequate first aid. Most countries have implemented some form of emergency services in order to do first aid in emergency situation like traffic accident. In Bandung city, emergency services used 112 as a single emergency call and managed by operators. These operators have to decide which unit is nearest to emergency location. The environment they operate is characterized by high degree of uncertainty, as they don't know exact distance from each unit to emergency location. K-nearest neighbor (KNN) can be used to finding nearest emergency unit. However, with the high number of emergency calls in Bandung city, KNN has high computational cost. It has to measuring each emergency unit distance to emergency location over and over. With Network Voronoi diagram (NVD) as extension, KNN's role can be optimized to find nearest emergency units by partitioning space into territories for each unit. So NVD can directly gets the nearest unit by looking into which unit's territory emergency location exist.

\vspace{0.5 cm}
\begin{flushleft}
{\textbf{Keywords:} Emergency, First aid, NVD, KNN.}
\end{flushleft}